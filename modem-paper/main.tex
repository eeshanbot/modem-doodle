\title[JASA/Sample JASA Article]{Sample JASA Article}
\author{EeShan Bhatt}
\author{Oscar Viquez}
\author{Henrik Schmidt}
\affiliation{Department of Mechanical Engineering,  Massachusetts Institute of Technology, Cambridge, MA 02139, USA}


% \author{Author Four}
% \email{author.four@university.edu}
% \affiliation{Department2,  University2, City, State ZipCode, Country}

% \author{Author Five}
% \altaffiliation{Also at: Department, University, City, State ZipCode, Country.}
% \affiliation{Department3,  University3, City, State ZipCode, Country}


\preprint{Bhatt, JASA}   %  if you want want this message to appear in upper right corner of title page

\date{\today}

\begin{abstract}
Put your abstract here. Abstracts are limited to 200 words for
regular articles and 100 words for Letters to the Editor. Please no
personal pronouns, also please do not use the words ``new'' and/or
``novel'' in the abstract. An article usually includes an abstract, a
concise summary of the work covered at length in the main body of the
article.
\end{abstract}

%% pacs numbers not used

\maketitle

%  End of title page for Preprint option --------------------------------- %



\section{\label{sec:1} Introduction}
This sample document demonstrates the use of JASA in manuscripts
prepared for submission to the Journal of the Acoustical Society of America.

The paper is organized as follows: Section~\ref{sec:2} presents
initial information, while
Section~\ref{sec:3} presents examples of mathematical expressions.

 % Sample showing how to include figure, this is a floating figure

\begin{figure}[ht]
%% \reprintcolumnwidth is the same in preprint and reprint for
%% ease of use for authors:
\includegraphics[width=\reprintcolumnwidth]{figsamp.jpg}
\caption{\label{fig:FIG1}{Caption here.}}

\raggedright
Note: The only figure formats allowed are the following:
.pdf, .ps, .eps, or .jpg. Figure files must be named in this fashion:
Figure\#.xxx, where ``\#'' is the figure number and ``xxx'' is the file format
(Figure1.eps, Figure2.jpg, Figure3a.ps, Figure3b.ps, etc).

[For these sample pages we have used only figsamp.jpg for convenience]
\end{figure}

\section{\label{sec:2} Section Two}

An example of another first-level Section with following example text that refers to subsections using
\verb+\ref{subsec:XXX}+ ...  EXAMPLE: Some background in
section~\ref{sec:2} and details  in subsection~\ref{subsec:2:1}.

\subsection{\label{subsec:2:1} Sample subsection}


   \subsubsection{Sample subsubsection\label{subsubsec:1}}

\paragraph{Sample paragraph}Here is text following the paragraph
heading.
Here is a figure reference: is shown in Fig.~\ref{fig:FIG1}.



Normal journal cite: \citep{joursamp1},
 Book reference \citet{booksamp1},
Computer language documentation:
\citep{sampcode2}.

Every \verb+\citep+  or \verb+\citet{}+ will produce a citation and an entry in the
bibliography. Every citation must have a matching entry in the
bibliography
database file (\verb+\filename.bib+).

Make your bibliography by doing: pdflatex filename,  bibtex filename,
pdflatex filename, pdflatex filename.

{\bfseries\itshape
Compare the results you get with\\
{\verb+\documentclass[preprint]{JASA}+ }\\
vs.\\
{\verb+\documentclass[preprint,NumberedRefs]{JASA}+ }
}

\bibliography{sampbib}
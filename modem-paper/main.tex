\title[JASA/Sample JASA Article]{Considering vertical sound speed uncertainty in the Beaufort Lens for more realistic range estimation between acoustic modems}
\author{EeShan Bhatt}
\author{Oscar Viquez}
\author{Bradli Howard}
\author{Henrik Schmidt}
\affiliation{Department of Mechanical Engineering,  Massachusetts Institute of Technology, Cambridge, MA 02139, USA}

% \author{Author Four}
% \email{author.four@university.edu}
% \affiliation{Department2,  University2, City, State ZipCode, Country}


\preprint{Bhatt, JASA}   %  if you want want this message to appear in upper right corner of title page

\date{\today}

\begin{abstract}

{\color{red}[Draft Abstract]}

Put your abstract here. Abstracts are limited to 200 words for
regular articles and 100 words for Letters to the Editor. Please no
personal pronouns, also please do not use the words ``new'' and/or
``novel'' in the abstract. An article usually includes an abstract, a
concise summary of the work covered at length in the main body of the
article.
\end{abstract}

%% pacs numbers not used

\maketitle



% =========================================================================== %
% =========================================================================== %
\section{\label{sec:1} Introduction}

{\color{red}
[Intro : EOF Approach, generally]

[Motivation: ICEX (\& Irminger?)]
}

This sample document demonstrates the use of JASA in manuscripts
prepared for submission to the Journal of the Acoustical Society of America.

The paper is organized as follows: Section~\ref{sec:2} presents
initial information, while
Section~\ref{sec:3} presents examples of mathematical expressions.



% =========================================================================== %
% =========================================================================== %
\section{\label{sec:2} Methods}

\subsection{\label{subsec:2:1} Environmental EOFs}

{\color{red}[DETAIL : EOF Approach]}

% ================================================== %
\begin{figure}[ht]
\figline{\fig{figsamp.jpg}{\reprintcolumnwidth}{(a)}}
\figline{\fig{figsamp.jpg}{4cm}{(b)} \fig{figsamp.jpg}{4cm}{(c)}}

% \includegraphics[width=\reprintcolumnwidth]{figsamp.jpg}
\caption{\label{fig:FIG1}{Extracting EOFs from environmental data: (a) HYCOM and ITP data, (b) EOF shapes and weight distributions.}}

\end{figure}
% ================================================== %

This section would generally discuss EOFs, the equations that extract them from data collections, and perhaps the "information capacity" w.r.t. to the number of depth points vs number of samples. Should we also include, in somewhat general terms, the "uniqueness" constraint? As in, i.e.: sampled region is too large or too small to adequately capture the features needed for reconstruction.



% =========================================================================== %
% =========================================================================== %
\section{\label{sec:3} Results}

\subsection{\label{subsec:3:1} ICEX 2020 conditions}
{\color{red}[CTD comparison]}

% ================================================== %
\begin{figure}[ht]
\includegraphics[width=\reprintcolumnwidth]{figsamp.jpg}
\caption{\label{fig:FIG2}{CTD casts vs. HYCOM-derived baseline}}
\end{figure}
% ================================================== %

% ========================= %
% ========================= %
\subsection{\label{subsec:3:2} One-Way Travel Times}
{\color{red}[OWTT comparison]}

% ================================================== %
\begin{figure}[ht]
\includegraphics[width=\reprintcolumnwidth]{figsamp.jpg}
\caption{\label{fig:FIG3}{Modem placement diagram}}
\end{figure}
% ================================================== %

% ================================================== %
\begin{figure}[ht]
\figline{\fig{figsamp.jpg}{\reprintcolumnwidth}{(a)}}
\figline{\fig{figsamp.jpg}{4cm}{(b)} \fig{figsamp.jpg}{4cm}{(c)}}
% \includegraphics[width=\reprintcolumnwidth]{figsamp.jpg}
\caption{\label{fig:FIG4}{(a) OWTT histogram, (b) "..." OF RANGE "...".}}
\end{figure}
% ================================================== %



% =========================================================================== %
% =========================================================================== %
\section{\label{sec:4} Discussion}

\subsection{\label{subsec:4:1} Model agreement discussion}
{\color{red}[Model agreement discussion]}

% ================================================== %
\begin{figure}[ht]
\includegraphics[width=\reprintcolumnwidth]{figsamp.jpg}
\caption{\label{fig:FIG5}{Comparison of range-independent model vs. field data}}
\end{figure}
% ================================================== %

\subsection{\label{subsec:4:2} Bootstrapping EOF}
{\color{red}[Bootstrapping EOF]}

% ================================================== %
\begin{figure}[ht]
\includegraphics[width=\reprintcolumnwidth]{figsamp.jpg}
\caption{\label{fig:FIG6}{Convergence}}
\end{figure}
% ================================================== %

% ================================================== %
\begin{figure}[ht]
\includegraphics[width=\reprintcolumnwidth]{figsamp.jpg}
\caption{\label{fig:FIG7}{Bootstrap error estimate}}
\end{figure}
% ================================================== %







% ================================================== %
\begin{figure}[ht]
\includegraphics[width=\reprintcolumnwidth]{figsamp.jpg}
\caption{\label{fig:FIG1}{Caption here.}}

\raggedright
{\color{red}
Note: The only figure formats allowed are the following:
.pdf, .ps, .eps, or .jpg. Figure files must be named in this fashion:
Figure\#.xxx, where ``\#'' is the figure number and ``xxx'' is the file format
(Figure1.eps, Figure2.jpg, Figure3a.ps, Figure3b.ps, etc).
}

[For these sample pages we have used only figsamp.jpg for convenience]
\end{figure}
% ================================================== %


An example of another first-level Section with following example text that refers to subsections using
\verb+\ref{subsec:XXX}+ ...  EXAMPLE: Some background in
section~\ref{sec:2} and details  in subsection~\ref{subsec:2:1}.



   \subsubsection{Sample subsubsection\label{subsubsec:1}}

\paragraph{Sample paragraph}Here is text following the paragraph
heading.
Here is a figure reference: is shown in Fig.~\ref{fig:FIG1}.



Normal journal cite: \citep{joursamp1},
 Book reference \citet{booksamp1},
Computer language documentation:
\citep{sampcode2}.

Every \verb+\citep+  or \verb+\citet{}+ will produce a citation and an entry in the
bibliography. Every citation must have a matching entry in the
bibliography
database file (\verb+\filename.bib+).


Here is an example of algorithmic:


\begin{algorithmic}
\If {$i\geq maxval$}
    \State $i\gets 0$
\Else
    \If {$i+k\leq maxval$}
        \State $i\gets i+k$
    \EndIf
\EndIf
\end{algorithmic}


\begin{acknowledgments}
This research was supported by  ...
\end{acknowledgments}


\bibliography{sampbib}